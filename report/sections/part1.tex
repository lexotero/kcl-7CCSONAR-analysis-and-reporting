\section{Part 1 - Data and Analysis Methodologies}

\subsection{Source of Data}

For our research, we selected a dataset containing 1000 records of patients from a multispecialty hospital in India.
Each record contains the patient ID, age and gender, and 12 other features representing the common observations and
information taken for patients presenting heart disease symptoms. All records are annotated showing whether the patient
suffers or not from a heart disease. Table \ref{description-of-features} shows a detail description of all the features
in the dataset.

\small
\begin{tabularx}{\linewidth}{ | X | X | X | X |}
    \caption{Description of features}\label{description-of-features} \\
    \hline
    \textbf{Feature} & \textbf{Code} & \textbf{Description} & \textbf{Values} \\
    \hline
    Patient ID
    & patientid
    & The patient's identification number.
    & A number.\\
    \hline
    Age
    & age
    & The age of the patient.
    & The age in years.\\
    \hline
    Gender
    & gender
    & The gender of the patient.
    & 0 (female) or 1 (male).\\
    \hline
    Resting blood pressure
    & restingBP
    & The blood pressure measured at rest.
    & 94-200 mm/HG.\\
    \hline
    Serum cholesterol
    & serumcholestrol
    & The levels of HDL and LDL plus 20\% the levels of triglycerides.
    & 126-564 mg/dl.\\
    \hline
    Fasting blood sugar
    & fastingbloodsugar
    & Whether the basal sugar levels are above 120 mg/dl or not.
    & 0 (no, < 120 mg/dl) or 1 (yes, > 120 mg/dl).\\
    \hline
    Chest pain type
    & chestpain
    & The type of chest pain reported by the patient.
    & 0 (typical angina), 1 (atypical angina), 2 (non-anginal pain) or 3 (asymptomatic).\\
    \hline
    Resting electrocardiogram results
    & restingelectro
    & Classification of the results of an electrocardiogram performed at rest.
    & 0 (normal), 1 (ST-T wave abnormality), 2 (probable or definite left ventricular hypertrophy).\\
    \hline
    Maximum heart rate achieved
    & maxheartrate
    & Maximum heart rate registered while performing light exercise.
    & 71-202 BPM.\\
    \hline
    Exercise-induced angina
    & exerciseangina
    & Whether the exercise induced an angina.
    & 0 (no) or 1 (yes).\\
    \hline
    Oldpeak
    & oldpeak
    & ST depression induced by exercise to rest.
    & 0-6.2\\
    \hline
    Slope
    & slope
    & The slope of the peak exercise ST segment.
    & 1 (upsloping), 2 (flat) or 3 (downsloping).\\
    \hline
    Number of major vessels
    & noofmajorvessels
    & Major vessels coloured by fluoroscopy.
    & A count (0-3).\\
    \hline
    Classification
    & target
    & Whether the patient suffers from a heart disease.
    & 0 (no) or 1 (yes).\\
    \hline
\end{tabularx}
\normalsize

Although it is not directly specified in the files distributed with the dataset, the features:

\begin{itemize}
    \item Resting blood pressure.
    \item Resting electrocardiogram results.
    \item Maximum heart rate achieved.
    \item Oldpeak.
    \item Slope.
    \item Exercise-induced angina.
\end{itemize}

are obtained by subjecting patients to a protocol of a rest period, followed by a period of light-to-moderate
cardiovascular exercise, finally followed by another rest period. This information is important to contextualise
the values found in the dataset.

The dataset is publicly available in Kaggle. \href{https://www.kaggle.com/}{Kaggle} is the largest online platform for data scientists, providing both
tools and data for scientific analysis, as well as hosting competitions. It is currently the biggest and
most active online scientific community, with over 15 million users \cite{Kaggle}.

\subsection{Limitations}

We chose this dataset as it contained common features used in research for cardiovascular disease prediction \cite{STSegmentChanges}
\cite{STSlopePredictor} \cite{STSlopePredictorComparison}. Nonetheless, we recognise that there is not much information available
about the dataset and how it was collected. The author has good reputation in Kaggle, but without a proper trace for the origin of
the records, we should be sceptical about the results, particularly about the chosen prediction model.
We mitigate this limitation by applying our prediction model to a well documented dataset, the heart disease dataset maintained by
UC Irvine \cite{HeartDiseaseDataset}.

\subsection{Research Objectives and Analysis Methodologies}

In this section, we introduce the research objectives and the selected analysis methodologies applied to the dataset
to meant these objectives.

This research has a main objective and three "sub-objectives" that help us understand the data we are working with
and expand our knowledge of the field.

\subsubsection{Main Objective: Generate a prediction model}

The main objective of this report is to generate a prediction model that can accurately label patients with heart disease
for a set of values for the given features. Having this prediction model would allow medical professionals to automatically
diagnose patients even helping prevent the development of cardiovascular diseases thanks to early diagnosis.

To generate a prediction model, we will try different well-known classification algorithms, such as K-Nearest Neighbour, 
Decision Trees, and Random Forest. For each of the evaluated algorithms, we will train the model using the dataset and
generate accuracy markers for the resulting predictions.

Once we have identified a classification algorithm that yields good predictions (we set the threshold at >95\%), we can
use it to generate a prediction for different dataset, thus verifying its accuracy and ensuring that we are not over-fitting.

\subsubsection{Sub-objective 1: Understand demographic distribution}

One of the first analysis we must carry out against the dataset is a demographic analysis. We want to understand
the demographic distribution so that we can identify potential biases (e.g. age bias and gender bias).

It is well known that, in the past, medical studies had a high incidence of gender bias, particularly in cardiovascular
research \cite{GenderBiasCardiovascularResearch}. For this reason, we must make sure that the dataset population
is well balanced to ensure that our prediction model will work for any age and gender.

To meet this objective we used frequency analysis. By grouping patients by age and gender and calculating frequency
distributions, we can quickly identify unbalanced populations (of a particular gender or age group).

\subsubsection{Sub-objective 2: Identify potential predictors in categorical data}

Another objective of our analysis is to identify potential predictors of heart disease in categorical data. This will
increase our knowledge of the field, and help us direct further research to specific features. In particular, we want
to be able to identify correlations between disease and categorical values.

Because of the nature of categorical data, the best analysis methodology in this case is again frequency analysis.
We group the records by age group and gender and provide percentage values for each of the features for the
identified groups.

\subsubsection{Sub-objective 3: Identify potential predictors in numerical data}

Lastly, we want to identify potential predictors of heart disease in the numerical data, again increasing our
knowledge of the field, and highlighting the features that have the highest correlation to the presence of
heart disease.

In this case, because we are dealing with numeric data, we can apply measures of central tendency, and measures
of dispersion. For these features, we produce box plots and tables with mean, median, percentiles, min, max, and
standard deviation.

Finally, we also generate a correlation matrix (which also includes categorical values) to properly calculate
any potential correlation. To help reading this correlation matrix, we use a heatmap and scatter diagrams.

