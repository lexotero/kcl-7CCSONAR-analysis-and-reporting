\section{Part 4 - Findings and Conclusions}

When carrying out this report, we set ourselves the following objectives:

\begin{itemize}
    \item Main Objective: Generate a prediction model.
    \item Sub-objective 1: Understand demographic distribution.
    \item Sub-objective 2: Identify potential predictors in categorical data.
    \item Sub-objective 3: Identify potential predictors in numerical data
\end{itemize}

By producing measures of central tendency, measures of dispersion, and analysing frequencies we've managed 
to gain a deep understanding of the dataset and its characteristics. Particularly, we've been able to show
that:

\begin{itemize}
    \item There is gender bias where females are under-represented, although the feature distribution are balanced across age and gender.
    \item All of the categorical features a degree of correlation to the presence of heart disease.
    \item The "Slope" feature has the highest degree of correlation and represents the main indicator of disease.
    \item There are significant differences in the measurements of central tendency and dispersion between genders for all numerical features. This highlights the need for a less biased dataset.
\end{itemize}

Using well-known classification algorithms we have also satisfied the main objective of our research. We have
identified that the Random Forest classification algorithm can be used to produce accurate (99\%) predictions.

Overall, we believe that the information contained in this report and particularly the prediction model could
be used to assist medical professionals with diagnosis using standard test that are already commonly performed.